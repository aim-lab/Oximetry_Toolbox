%% Generated by Sphinx.
\def\sphinxdocclass{report}
\documentclass[letterpaper,10pt,english]{sphinxmanual}
\ifdefined\pdfpxdimen
   \let\sphinxpxdimen\pdfpxdimen\else\newdimen\sphinxpxdimen
\fi \sphinxpxdimen=.75bp\relax

\PassOptionsToPackage{warn}{textcomp}
\usepackage[utf8]{inputenc}
\ifdefined\DeclareUnicodeCharacter
% support both utf8 and utf8x syntaxes
  \ifdefined\DeclareUnicodeCharacterAsOptional
    \def\sphinxDUC#1{\DeclareUnicodeCharacter{"#1}}
  \else
    \let\sphinxDUC\DeclareUnicodeCharacter
  \fi
  \sphinxDUC{00A0}{\nobreakspace}
  \sphinxDUC{2500}{\sphinxunichar{2500}}
  \sphinxDUC{2502}{\sphinxunichar{2502}}
  \sphinxDUC{2514}{\sphinxunichar{2514}}
  \sphinxDUC{251C}{\sphinxunichar{251C}}
  \sphinxDUC{2572}{\textbackslash}
\fi
\usepackage{cmap}
\usepackage[T1]{fontenc}
\usepackage{amsmath,amssymb,amstext}
\usepackage{babel}



\usepackage{times}
\expandafter\ifx\csname T@LGR\endcsname\relax
\else
% LGR was declared as font encoding
  \substitutefont{LGR}{\rmdefault}{cmr}
  \substitutefont{LGR}{\sfdefault}{cmss}
  \substitutefont{LGR}{\ttdefault}{cmtt}
\fi
\expandafter\ifx\csname T@X2\endcsname\relax
  \expandafter\ifx\csname T@T2A\endcsname\relax
  \else
  % T2A was declared as font encoding
    \substitutefont{T2A}{\rmdefault}{cmr}
    \substitutefont{T2A}{\sfdefault}{cmss}
    \substitutefont{T2A}{\ttdefault}{cmtt}
  \fi
\else
% X2 was declared as font encoding
  \substitutefont{X2}{\rmdefault}{cmr}
  \substitutefont{X2}{\sfdefault}{cmss}
  \substitutefont{X2}{\ttdefault}{cmtt}
\fi


\usepackage[Bjarne]{fncychap}
\usepackage{sphinx}

\fvset{fontsize=\small}
\usepackage{geometry}


% Include hyperref last.
\usepackage{hyperref}
% Fix anchor placement for figures with captions.
\usepackage{hypcap}% it must be loaded after hyperref.
% Set up styles of URL: it should be placed after hyperref.
\urlstyle{same}

\addto\captionsenglish{\renewcommand{\contentsname}{Contents:}}

\usepackage{sphinxmessages}
\setcounter{tocdepth}{3}
\setcounter{secnumdepth}{3}


\title{pobm Documentation}
\date{Oct 18, 2020}
\release{}
\author{Author}
\newcommand{\sphinxlogo}{\vbox{}}
\renewcommand{\releasename}{}
\makeindex
\begin{document}

\ifdefined\shorthandoff
  \ifnum\catcode`\=\string=\active\shorthandoff{=}\fi
  \ifnum\catcode`\"=\active\shorthandoff{"}\fi
\fi

\pagestyle{empty}
\sphinxmaketitle
\pagestyle{plain}
\sphinxtableofcontents
\pagestyle{normal}
\phantomsection\label{\detokenize{index::doc}}



\chapter{pobm package}
\label{\detokenize{pobm:pobm-package}}\label{\detokenize{pobm::doc}}

\section{Subpackages}
\label{\detokenize{pobm:subpackages}}

\subsection{pobm.obm package}
\label{\detokenize{pobm.obm:pobm-obm-package}}\label{\detokenize{pobm.obm::doc}}

\subsubsection{Submodules}
\label{\detokenize{pobm.obm:submodules}}

\subsubsection{pobm.obm.burden}
\label{\detokenize{pobm.obm:module-pobm.obm.burden}}\label{\detokenize{pobm.obm:pobm-obm-burden}}\index{module@\spxentry{module}!pobm.obm.burden@\spxentry{pobm.obm.burden}}\index{pobm.obm.burden@\spxentry{pobm.obm.burden}!module@\spxentry{module}}\index{HypoxicBurdenMeasures (class in pobm.obm.burden)@\spxentry{HypoxicBurdenMeasures}\spxextra{class in pobm.obm.burden}}

\begin{fulllineitems}
\phantomsection\label{\detokenize{pobm.obm:pobm.obm.burden.HypoxicBurdenMeasures}}\pysiglinewithargsret{\sphinxbfcode{\sphinxupquote{class }}\sphinxcode{\sphinxupquote{pobm.obm.burden.}}\sphinxbfcode{\sphinxupquote{HypoxicBurdenMeasures}}}{\emph{\DUrole{n}{begin}}, \emph{\DUrole{n}{end}}, \emph{\DUrole{n}{CT\_Threshold}\DUrole{o}{=}\DUrole{default_value}{90}}, \emph{\DUrole{n}{CA\_Baseline}\DUrole{o}{=}\DUrole{default_value}{None}}}{}
Bases: \sphinxcode{\sphinxupquote{object}}

Class that calculates Hypoxic Burden Features from spo2 time series.
Suppose that the data has been preprocessed.
\begin{quote}\begin{description}
\item[{Parameters}] \leavevmode\begin{itemize}
\item {} 
\sphinxstyleliteralstrong{\sphinxupquote{begin}} \textendash{} List of indices of beginning of each desaturation event.

\item {} 
\sphinxstyleliteralstrong{\sphinxupquote{end}} \textendash{} List of indices of end of each desaturation event.

\item {} 
\sphinxstyleliteralstrong{\sphinxupquote{CT\_Threshold}} \textendash{} Percentage of the time spent below the “CT\_Threshold” \% oxygen saturation level.

\item {} 
\sphinxstyleliteralstrong{\sphinxupquote{CA\_Baseline}} \textendash{} Baseline to compute the CA feature. Default value is mean of the signal.

\end{itemize}

\end{description}\end{quote}
\index{compute() (pobm.obm.burden.HypoxicBurdenMeasures method)@\spxentry{compute()}\spxextra{pobm.obm.burden.HypoxicBurdenMeasures method}}

\begin{fulllineitems}
\phantomsection\label{\detokenize{pobm.obm:pobm.obm.burden.HypoxicBurdenMeasures.compute}}\pysiglinewithargsret{\sphinxbfcode{\sphinxupquote{compute}}}{\emph{\DUrole{n}{signal}}}{}~\begin{quote}\begin{description}
\item[{Parameters}] \leavevmode
\sphinxstyleliteralstrong{\sphinxupquote{signal}} \textendash{} 1\sphinxhyphen{}d array, of shape (N,) where N is the length of the signal

\item[{Returns}] \leavevmode
\begin{description}
\item[{HypoxicBurdenMeasuresResults class containing the following features:}] \leavevmode\begin{itemize}
\item {} 
CA: Integral SpO2 below the xx SpO2 level normalized by the total recording time

\item {} 
CT: Percentage of the time spent below the xx\% oxygen saturation level

\item {} 
POD: Percentage of oxygen desaturation events

\item {} 
AODmax: The area under the oxygen desaturation event curve, using the maximum SpO2 value as baseline
and normalized by the total recording time

\item {} 
AOD100: Cumulative area of desaturations under the 100\% SpO2 level as baseline and normalized
by the total recording time

\end{itemize}

\end{description}


\end{description}\end{quote}

Example:

\begin{sphinxVerbatim}[commandchars=\\\{\}]
\PYG{k+kn}{from} \PYG{n+nn}{pobm}\PYG{n+nn}{.}\PYG{n+nn}{obm}\PYG{n+nn}{.}\PYG{n+nn}{burden} \PYG{k+kn}{import} \PYG{n}{HypoxicBurdenMeasures}

\PYG{c+c1}{\PYGZsh{} Initialize the class with the desired parameters}
\PYG{n}{hypoxic\PYGZus{}class} \PYG{o}{=} \PYG{n}{HypoxicBurdenMeasures}\PYG{p}{(}\PYG{n}{results\PYGZus{}desat}\PYG{o}{.}\PYG{n}{begin}\PYG{p}{,} \PYG{n}{results\PYGZus{}desat}\PYG{o}{.}\PYG{n}{end}\PYG{p}{,} \PYG{n}{CT\PYGZus{}Threshold}\PYG{o}{=}\PYG{l+m+mi}{90}\PYG{p}{,} \PYG{n}{CA\PYGZus{}Baseline}\PYG{o}{=}\PYG{l+m+mi}{90}\PYG{p}{)}

\PYG{c+c1}{\PYGZsh{} Compute the biomarkers}
\PYG{n}{results\PYGZus{}hypoxic} \PYG{o}{=} \PYG{n}{hypoxic\PYGZus{}class}\PYG{o}{.}\PYG{n}{compute}\PYG{p}{(}\PYG{n}{spo2\PYGZus{}signal}\PYG{p}{)}
\end{sphinxVerbatim}

\end{fulllineitems}


\end{fulllineitems}



\subsubsection{pobm.obm.complex}
\label{\detokenize{pobm.obm:module-pobm.obm.complex}}\label{\detokenize{pobm.obm:pobm-obm-complex}}\index{module@\spxentry{module}!pobm.obm.complex@\spxentry{pobm.obm.complex}}\index{pobm.obm.complex@\spxentry{pobm.obm.complex}!module@\spxentry{module}}\index{ComplexityMeasures (class in pobm.obm.complex)@\spxentry{ComplexityMeasures}\spxextra{class in pobm.obm.complex}}

\begin{fulllineitems}
\phantomsection\label{\detokenize{pobm.obm:pobm.obm.complex.ComplexityMeasures}}\pysiglinewithargsret{\sphinxbfcode{\sphinxupquote{class }}\sphinxcode{\sphinxupquote{pobm.obm.complex.}}\sphinxbfcode{\sphinxupquote{ComplexityMeasures}}}{\emph{\DUrole{n}{CTM\_Threshold}\DUrole{o}{=}\DUrole{default_value}{0.25}}, \emph{\DUrole{n}{DFA\_Window}\DUrole{o}{=}\DUrole{default_value}{20}}, \emph{\DUrole{n}{M\_Sampen}\DUrole{o}{=}\DUrole{default_value}{3}}, \emph{\DUrole{n}{R\_Sampen}\DUrole{o}{=}\DUrole{default_value}{0.2}}, \emph{\DUrole{n}{M\_ApEn}\DUrole{o}{=}\DUrole{default_value}{2}}, \emph{\DUrole{n}{R\_ApEn}\DUrole{o}{=}\DUrole{default_value}{0.25}}}{}
Bases: \sphinxcode{\sphinxupquote{object}}

Class that calculates Complexity Features from spo2 time series.
Suppose that the data has been preprocessed.
\begin{description}
\item[{:param}] \leavevmode
signal: 1\sphinxhyphen{}d array, of shape (N,) where N is the length of the signal
CTM\_Threshold: Radius of Central Tendency Measure.
DFA\_Window: Length of window to calculate DFA biomarker.
M\_Sampen: Embedding dimension to compute SampEn.
R\_Sampen: Tolerance to compute SampEn.
M\_ApEn: Embedding dimension to compute ApEn.
R\_ApEn: Tolerance to compute ApEn.

\end{description}
\index{compute() (pobm.obm.complex.ComplexityMeasures method)@\spxentry{compute()}\spxextra{pobm.obm.complex.ComplexityMeasures method}}

\begin{fulllineitems}
\phantomsection\label{\detokenize{pobm.obm:pobm.obm.complex.ComplexityMeasures.compute}}\pysiglinewithargsret{\sphinxbfcode{\sphinxupquote{compute}}}{\emph{\DUrole{n}{signal}}}{{ $\rightarrow$ pobm.\_ResultsClasses.ComplexityMeasuresResults}}~\begin{quote}\begin{description}
\item[{Parameters}] \leavevmode
\sphinxstyleliteralstrong{\sphinxupquote{signal}} \textendash{} 1\sphinxhyphen{}d array, of shape (N,) where N is the length of the signal

\item[{Returns}] \leavevmode
\begin{description}
\item[{ComplexityMeasuresResults class containing the following features:}] \leavevmode\begin{itemize}
\item {} 
ApEn: Approximate Entropy.

\item {} 
LZ: Lempel\sphinxhyphen{}Ziv complexity.

\item {} 
CTM: Central Tendency Measure.

\item {} 
SampEn: Sample Entropy.

\item {} 
DFA: Detrended Fluctuation Analysis.

\end{itemize}

\end{description}


\end{description}\end{quote}

Example:

\begin{sphinxVerbatim}[commandchars=\\\{\}]
\PYG{k+kn}{from} \PYG{n+nn}{pobm}\PYG{n+nn}{.}\PYG{n+nn}{obm}\PYG{n+nn}{.}\PYG{n+nn}{complex} \PYG{k+kn}{import} \PYG{n}{ComplexityMeasures}

\PYG{c+c1}{\PYGZsh{} Initialize the class with the desired parameters}
\PYG{n}{complexity\PYGZus{}class} \PYG{o}{=} \PYG{n}{ComplexityMeasures}\PYG{p}{(}\PYG{n}{CTM\PYGZus{}Threshold}\PYG{o}{=}\PYG{l+m+mf}{0.25}\PYG{p}{,} \PYG{n}{DFA\PYGZus{}Window}\PYG{o}{=}\PYG{l+m+mi}{20}\PYG{p}{,} \PYG{n}{M\PYGZus{}Sampen}\PYG{o}{=}\PYG{l+m+mi}{3}\PYG{p}{,} \PYG{n}{R\PYGZus{}Sampen}\PYG{o}{=}\PYG{l+m+mf}{0.2}\PYG{p}{,} \PYG{n}{M\PYGZus{}ApEn}\PYG{o}{=}\PYG{l+m+mi}{2}\PYG{p}{,} \PYG{n}{R\PYGZus{}ApEn}\PYG{o}{=}\PYG{l+m+mf}{0.25}\PYG{p}{)}

\PYG{c+c1}{\PYGZsh{} Compute the biomarkers}
\PYG{n}{results\PYGZus{}complexity} \PYG{o}{=} \PYG{n}{complexity\PYGZus{}class}\PYG{o}{.}\PYG{n}{compute}\PYG{p}{(}\PYG{n}{spo2\PYGZus{}signal}\PYG{p}{)}
\end{sphinxVerbatim}

\end{fulllineitems}


\end{fulllineitems}



\subsubsection{pobm.obm.desat}
\label{\detokenize{pobm.obm:module-pobm.obm.desat}}\label{\detokenize{pobm.obm:pobm-obm-desat}}\index{module@\spxentry{module}!pobm.obm.desat@\spxentry{pobm.obm.desat}}\index{pobm.obm.desat@\spxentry{pobm.obm.desat}!module@\spxentry{module}}\index{DesaturationsMeasures (class in pobm.obm.desat)@\spxentry{DesaturationsMeasures}\spxextra{class in pobm.obm.desat}}

\begin{fulllineitems}
\phantomsection\label{\detokenize{pobm.obm:pobm.obm.desat.DesaturationsMeasures}}\pysiglinewithargsret{\sphinxbfcode{\sphinxupquote{class }}\sphinxcode{\sphinxupquote{pobm.obm.desat.}}\sphinxbfcode{\sphinxupquote{DesaturationsMeasures}}}{\emph{\DUrole{n}{ODI\_Threshold}\DUrole{o}{=}\DUrole{default_value}{3}}}{}
Bases: \sphinxcode{\sphinxupquote{object}}

Class that calculates the Desaturation Features from spo2 time series.
Suppose that the data has been preprocessed.
\begin{quote}\begin{description}
\item[{Parameters}] \leavevmode
\sphinxstyleliteralstrong{\sphinxupquote{ODI\_Threshold}} \textendash{} Threshold to compute Oxygen Desaturation Index.

\end{description}\end{quote}
\index{compute() (pobm.obm.desat.DesaturationsMeasures method)@\spxentry{compute()}\spxextra{pobm.obm.desat.DesaturationsMeasures method}}

\begin{fulllineitems}
\phantomsection\label{\detokenize{pobm.obm:pobm.obm.desat.DesaturationsMeasures.compute}}\pysiglinewithargsret{\sphinxbfcode{\sphinxupquote{compute}}}{\emph{\DUrole{n}{signal}}}{{ $\rightarrow$ pobm.\_ResultsClasses.DesaturationsMeasuresResults}}~\begin{quote}\begin{description}
\item[{Parameters}] \leavevmode
\sphinxstyleliteralstrong{\sphinxupquote{signal}} \textendash{} 1\sphinxhyphen{}d array, of shape (N,) where N is the length of the signal

\item[{Returns}] \leavevmode
\begin{description}
\item[{DesaturationsMeasuresResults class containing the following features:}] \leavevmode\begin{itemize}
\item {} 
DL\_u: Mean of desaturation length

\item {} 
DL\_sd: Standard deviation of desaturation length

\item {} 
DA100\_u: Mean of desaturation area using 100\% as baseline.

\item {} 
DA100\_sd: Standard deviation of desaturation area using 100\% as baseline

\item {} 
DAmax\_u: Mean of desaturation area using max value as baseline.

\item {} 
DAmax\_sd: Standard deviation of desaturation area using max value as baseline

\item {} 
DD100\_u: Mean of depth desaturation from 100\%.

\item {} 
DD100\_sd: Standard deviation of depth desaturation from 100\%.

\item {} 
DDmax\_u: Mean of depth desaturation from max value.

\item {} 
DDmax\_sd: Standard deviation of depth desaturation from max value.

\item {} 
DS\_u: Mean of the desaturation slope.

\item {} 
DS\_sd: Standard deviation of the desaturation slope.

\item {} 
TD\_u: Mean of time between two consecutive desaturation events.

\item {} 
TD\_sd: Standard deviation of time between 2 consecutive desaturation events.

\end{itemize}

\end{description}


\end{description}\end{quote}

Example:

\begin{sphinxVerbatim}[commandchars=\\\{\}]
\PYG{k+kn}{from} \PYG{n+nn}{pobm}\PYG{n+nn}{.}\PYG{n+nn}{obm}\PYG{n+nn}{.}\PYG{n+nn}{desat} \PYG{k+kn}{import} \PYG{n}{DesaturationsMeasures}

\PYG{c+c1}{\PYGZsh{} Initialize the class with the desired parameters}
\PYG{n}{desat\PYGZus{}class} \PYG{o}{=} \PYG{n}{DesaturationsMeasures}\PYG{p}{(}\PYG{n}{ODI\PYGZus{}Threshold}\PYG{o}{=}\PYG{l+m+mi}{3}\PYG{p}{)}

\PYG{c+c1}{\PYGZsh{} Compute the biomarkers}
\PYG{n}{results\PYGZus{}desat} \PYG{o}{=} \PYG{n}{desat\PYGZus{}class}\PYG{o}{.}\PYG{n}{compute}\PYG{p}{(}\PYG{n}{spo2\PYGZus{}signal}\PYG{p}{)}
\end{sphinxVerbatim}

\end{fulllineitems}

\index{desaturation\_detector() (pobm.obm.desat.DesaturationsMeasures method)@\spxentry{desaturation\_detector()}\spxextra{pobm.obm.desat.DesaturationsMeasures method}}

\begin{fulllineitems}
\phantomsection\label{\detokenize{pobm.obm:pobm.obm.desat.DesaturationsMeasures.desaturation_detector}}\pysiglinewithargsret{\sphinxbfcode{\sphinxupquote{desaturation\_detector}}}{\emph{\DUrole{n}{signal}}}{}
run desaturation detector, implemented by Dr. Joachim Behar
\begin{quote}\begin{description}
\item[{Parameters}] \leavevmode
\sphinxstyleliteralstrong{\sphinxupquote{signal}} \textendash{} The SpO2 signal, of shape (N,)

\item[{Returns}] \leavevmode
\begin{description}
\item[{ODIMeasureResult class containing the following features:}] \leavevmode\begin{itemize}
\item {} 
ODI: the average number of desaturation events per hour.

\item {} 
begin: List of indices of beginning of each desaturation event.

\item {} 
end: List of indices of end of each desaturation event.

\end{itemize}

\end{description}


\end{description}\end{quote}

\end{fulllineitems}


\end{fulllineitems}

\index{desat\_embedding() (in module pobm.obm.desat)@\spxentry{desat\_embedding()}\spxextra{in module pobm.obm.desat}}

\begin{fulllineitems}
\phantomsection\label{\detokenize{pobm.obm:pobm.obm.desat.desat_embedding}}\pysiglinewithargsret{\sphinxcode{\sphinxupquote{pobm.obm.desat.}}\sphinxbfcode{\sphinxupquote{desat\_embedding}}}{\emph{\DUrole{n}{begin}}, \emph{\DUrole{n}{end}}}{}
Help function for the class
\begin{quote}\begin{description}
\item[{Returns}] \leavevmode
helper arrays containing the information about desaturation lengths and areas.

\end{description}\end{quote}

\end{fulllineitems}



\subsubsection{pobm.obm.general}
\label{\detokenize{pobm.obm:module-pobm.obm.general}}\label{\detokenize{pobm.obm:pobm-obm-general}}\index{module@\spxentry{module}!pobm.obm.general@\spxentry{pobm.obm.general}}\index{pobm.obm.general@\spxentry{pobm.obm.general}!module@\spxentry{module}}\index{OverallGeneralMeasures (class in pobm.obm.general)@\spxentry{OverallGeneralMeasures}\spxextra{class in pobm.obm.general}}

\begin{fulllineitems}
\phantomsection\label{\detokenize{pobm.obm:pobm.obm.general.OverallGeneralMeasures}}\pysiglinewithargsret{\sphinxbfcode{\sphinxupquote{class }}\sphinxcode{\sphinxupquote{pobm.obm.general.}}\sphinxbfcode{\sphinxupquote{OverallGeneralMeasures}}}{\emph{\DUrole{n}{ZC\_Baseline}\DUrole{o}{=}\DUrole{default_value}{None}}, \emph{\DUrole{n}{percentile}\DUrole{o}{=}\DUrole{default_value}{1}}, \emph{\DUrole{n}{M\_Threshold}\DUrole{o}{=}\DUrole{default_value}{2}}, \emph{\DUrole{n}{DI\_Window}\DUrole{o}{=}\DUrole{default_value}{12}}}{}
Bases: \sphinxcode{\sphinxupquote{object}}

Class that calculates Overall General Features from spo2 time series.
Suppose that the data has been preprocessed.
\begin{quote}\begin{description}
\item[{Parameters}] \leavevmode\begin{itemize}
\item {} 
\sphinxstyleliteralstrong{\sphinxupquote{ZC\_Baseline}} \textendash{} Baseline for calculating number of zero\sphinxhyphen{}crossing points.

\item {} 
\sphinxstyleliteralstrong{\sphinxupquote{percentile}} \textendash{} Percentile to perform. For example, for percentile 1, the argument should be 1

\item {} 
\sphinxstyleliteralstrong{\sphinxupquote{M\_Threshold}} \textendash{} Percentage of the signal M\_Threshold \% below median oxygen saturation. Typically use 1,2 or 5

\end{itemize}

\end{description}\end{quote}
\index{compute() (pobm.obm.general.OverallGeneralMeasures method)@\spxentry{compute()}\spxextra{pobm.obm.general.OverallGeneralMeasures method}}

\begin{fulllineitems}
\phantomsection\label{\detokenize{pobm.obm:pobm.obm.general.OverallGeneralMeasures.compute}}\pysiglinewithargsret{\sphinxbfcode{\sphinxupquote{compute}}}{\emph{\DUrole{n}{signal}}}{{ $\rightarrow$ pobm.\_ResultsClasses.OverallGeneralMeasuresResult}}~\begin{quote}\begin{description}
\item[{Parameters}] \leavevmode
\sphinxstyleliteralstrong{\sphinxupquote{signal}} \textendash{} 1\sphinxhyphen{}d array, of shape (N,) where N is the length of the signal

\item[{Returns}] \leavevmode
\begin{description}
\item[{OveralGeneralMeasuresResult class containing the following features:}] \leavevmode\begin{itemize}
\item {} 
AV: Average of the signal.

\item {} 
MED: Median of the signal.

\item {} 
Min: Minimum value of the signal.

\item {} 
SD: Std of the signal.

\item {} 
RG: SpO2 range (difference between the max and min value).

\item {} 
P: percentile.

\item {} 
M: Percentage of the signal x\% below median oxygen saturation.

\item {} 
ZC: Number of zero\sphinxhyphen{}crossing points.

\item {} 
DI: Delta Index.

\end{itemize}

\end{description}


\end{description}\end{quote}

Example:

\begin{sphinxVerbatim}[commandchars=\\\{\}]
\PYG{k+kn}{from} \PYG{n+nn}{pobm}\PYG{n+nn}{.}\PYG{n+nn}{obm}\PYG{n+nn}{.}\PYG{n+nn}{general} \PYG{k+kn}{import} \PYG{n}{OverallGeneralMeasures}

\PYG{c+c1}{\PYGZsh{} Initialize the class with the desired parameters}
\PYG{n}{statistics\PYGZus{}class} \PYG{o}{=} \PYG{n}{OverallGeneralMeasures}\PYG{p}{(}\PYG{n}{ZC\PYGZus{}Baseline}\PYG{o}{=}\PYG{l+m+mi}{90}\PYG{p}{,} \PYG{n}{percentile}\PYG{o}{=}\PYG{l+m+mi}{1}\PYG{p}{,} \PYG{n}{M\PYGZus{}Threshold}\PYG{o}{=}\PYG{l+m+mi}{2}\PYG{p}{,} \PYG{n}{DI\PYGZus{}Window}\PYG{o}{=}\PYG{l+m+mi}{12}\PYG{p}{)}

\PYG{c+c1}{\PYGZsh{} Compute the biomarkers}
\PYG{n}{results\PYGZus{}statistics} \PYG{o}{=} \PYG{n}{statistics\PYGZus{}class}\PYG{o}{.}\PYG{n}{compute}\PYG{p}{(}\PYG{n}{spo2\PYGZus{}signal}\PYG{p}{)}
\end{sphinxVerbatim}

\end{fulllineitems}


\end{fulllineitems}



\subsubsection{pobm.obm.periodicity}
\label{\detokenize{pobm.obm:module-pobm.obm.periodicity}}\label{\detokenize{pobm.obm:pobm-obm-periodicity}}\index{module@\spxentry{module}!pobm.obm.periodicity@\spxentry{pobm.obm.periodicity}}\index{pobm.obm.periodicity@\spxentry{pobm.obm.periodicity}!module@\spxentry{module}}\index{PRSAMeasures (class in pobm.obm.periodicity)@\spxentry{PRSAMeasures}\spxextra{class in pobm.obm.periodicity}}

\begin{fulllineitems}
\phantomsection\label{\detokenize{pobm.obm:pobm.obm.periodicity.PRSAMeasures}}\pysiglinewithargsret{\sphinxbfcode{\sphinxupquote{class }}\sphinxcode{\sphinxupquote{pobm.obm.periodicity.}}\sphinxbfcode{\sphinxupquote{PRSAMeasures}}}{\emph{\DUrole{n}{PRSA\_Window}\DUrole{o}{=}\DUrole{default_value}{10}}, \emph{\DUrole{n}{K\_AC}\DUrole{o}{=}\DUrole{default_value}{2}}}{}
Bases: \sphinxcode{\sphinxupquote{object}}

Function that calculates PRSA Features from spo2 time series.
Suppose that the data has been preprocessed.
\begin{description}
\item[{:param}] \leavevmode
PRSA\_Window: Fragment duration of PRSA.
K\_AC: Number of values to shift when computing autocorrelation

\end{description}
\index{compute() (pobm.obm.periodicity.PRSAMeasures method)@\spxentry{compute()}\spxextra{pobm.obm.periodicity.PRSAMeasures method}}

\begin{fulllineitems}
\phantomsection\label{\detokenize{pobm.obm:pobm.obm.periodicity.PRSAMeasures.compute}}\pysiglinewithargsret{\sphinxbfcode{\sphinxupquote{compute}}}{\emph{\DUrole{n}{signal}}}{{ $\rightarrow$ pobm.\_ResultsClasses.PRSAResults}}~\begin{quote}\begin{description}
\item[{Parameters}] \leavevmode
\sphinxstyleliteralstrong{\sphinxupquote{signal}} \textendash{} 1\sphinxhyphen{}d array, of shape (N,) where N is the length of the signal

\item[{Returns}] \leavevmode
\begin{description}
\item[{PRSAResults class containing the following features:}] \leavevmode\begin{itemize}
\item {} 
PRSAc: PRSA capacity.

\item {} 
PRSAad: PRSA amplitude difference.

\item {} 
PRSAos: PRSA overall slope.

\item {} 
PRSAsb: PRSA slope before the anchor point.

\item {} 
PRSAsa: PRSA slope after the anchor point.

\item {} 
AC: Autocorrelation.

\end{itemize}

\end{description}


\end{description}\end{quote}

Example:

\begin{sphinxVerbatim}[commandchars=\\\{\}]
\PYG{k+kn}{from} \PYG{n+nn}{pobm}\PYG{n+nn}{.}\PYG{n+nn}{obm}\PYG{n+nn}{.}\PYG{n+nn}{periodicity} \PYG{k+kn}{import} \PYG{n}{PRSAMeasures}

\PYG{c+c1}{\PYGZsh{} Initialize the class with the desired parameters}
\PYG{n}{prsa\PYGZus{}class} \PYG{o}{=} \PYG{n}{PRSAMeasures}\PYG{p}{(}\PYG{n}{PRSA\PYGZus{}Window}\PYG{o}{=}\PYG{l+m+mi}{10}\PYG{p}{,} \PYG{n}{K\PYGZus{}AC}\PYG{o}{=}\PYG{l+m+mi}{2}\PYG{p}{)}

\PYG{c+c1}{\PYGZsh{} Compute the biomarkers}
\PYG{n}{results\PYGZus{}PRSA} \PYG{o}{=} \PYG{n}{prsa\PYGZus{}class}\PYG{o}{.}\PYG{n}{compute}\PYG{p}{(}\PYG{n}{spo2\PYGZus{}signal}\PYG{p}{)}
\end{sphinxVerbatim}

\end{fulllineitems}


\end{fulllineitems}

\index{PSDMeasures (class in pobm.obm.periodicity)@\spxentry{PSDMeasures}\spxextra{class in pobm.obm.periodicity}}

\begin{fulllineitems}
\phantomsection\label{\detokenize{pobm.obm:pobm.obm.periodicity.PSDMeasures}}\pysigline{\sphinxbfcode{\sphinxupquote{class }}\sphinxcode{\sphinxupquote{pobm.obm.periodicity.}}\sphinxbfcode{\sphinxupquote{PSDMeasures}}}
Bases: \sphinxcode{\sphinxupquote{object}}

Function that calculates PSD Features from spo2 time series.
Suppose that the data has been preprocessed.
\index{compute() (pobm.obm.periodicity.PSDMeasures method)@\spxentry{compute()}\spxextra{pobm.obm.periodicity.PSDMeasures method}}

\begin{fulllineitems}
\phantomsection\label{\detokenize{pobm.obm:pobm.obm.periodicity.PSDMeasures.compute}}\pysiglinewithargsret{\sphinxbfcode{\sphinxupquote{compute}}}{\emph{\DUrole{n}{signal}}}{{ $\rightarrow$ pobm.\_ResultsClasses.PSDResults}}~\begin{description}
\item[{:param}] \leavevmode
signal: The SpO2 signal, of shape (N,)

\end{description}
\begin{quote}\begin{description}
\item[{Returns}] \leavevmode
\begin{description}
\item[{PSDResults class containing the following features:}] \leavevmode\begin{itemize}
\item {} 
PSD\_total: The amplitude of the spectral signal.

\item {} 
PSD\_band: The amplitude of the signal multiplied by a band\sphinxhyphen{}pass filter between 0.014 and 0.033 Hz.

\item {} 
PSD\_ratio: The ratio between PSD\_total and PSD\_band.

\item {} 
PDS\_peak: The max value of the FFT into the band 0.014\sphinxhyphen{}0.033 Hz.

\end{itemize}

\end{description}


\end{description}\end{quote}

Example:

\begin{sphinxVerbatim}[commandchars=\\\{\}]
\PYG{k+kn}{from} \PYG{n+nn}{pobm}\PYG{n+nn}{.}\PYG{n+nn}{obm}\PYG{n+nn}{.}\PYG{n+nn}{periodicity} \PYG{k+kn}{import} \PYG{n}{PSDMeasures}

\PYG{c+c1}{\PYGZsh{} Initialize the class with the desired parameters}
\PYG{n}{psd\PYGZus{}class} \PYG{o}{=} \PYG{n}{PSDMeasures}\PYG{p}{(}\PYG{p}{)}

\PYG{c+c1}{\PYGZsh{} Compute the biomarkers}
\PYG{n}{results\PYGZus{}PSD} \PYG{o}{=} \PYG{n}{psd\PYGZus{}class}\PYG{o}{.}\PYG{n}{compute}\PYG{p}{(}\PYG{n}{spo2\PYGZus{}signal}\PYG{p}{)}
\end{sphinxVerbatim}

\end{fulllineitems}


\end{fulllineitems}



\subsubsection{Module contents}
\label{\detokenize{pobm.obm:module-pobm.obm}}\label{\detokenize{pobm.obm:module-contents}}\index{module@\spxentry{module}!pobm.obm@\spxentry{pobm.obm}}\index{pobm.obm@\spxentry{pobm.obm}!module@\spxentry{module}}

\subsection{pobm.spo2 package}
\label{\detokenize{pobm.spo2:pobm-spo2-package}}\label{\detokenize{pobm.spo2::doc}}

\subsubsection{Submodules}
\label{\detokenize{pobm.spo2:submodules}}

\subsubsection{pobm.spo2.single\_biomarkers}
\label{\detokenize{pobm.spo2:module-pobm.spo2.single_biomarkers}}\label{\detokenize{pobm.spo2:pobm-spo2-single-biomarkers}}\index{module@\spxentry{module}!pobm.spo2.single\_biomarkers@\spxentry{pobm.spo2.single\_biomarkers}}\index{pobm.spo2.single\_biomarkers@\spxentry{pobm.spo2.single\_biomarkers}!module@\spxentry{module}}\index{apen() (in module pobm.spo2.single\_biomarkers)@\spxentry{apen()}\spxextra{in module pobm.spo2.single\_biomarkers}}

\begin{fulllineitems}
\phantomsection\label{\detokenize{pobm.spo2:pobm.spo2.single_biomarkers.apen}}\pysiglinewithargsret{\sphinxcode{\sphinxupquote{pobm.spo2.single\_biomarkers.}}\sphinxbfcode{\sphinxupquote{apen}}}{\emph{\DUrole{n}{signal}}, \emph{\DUrole{n}{M\_ApEn}\DUrole{o}{=}\DUrole{default_value}{2}}, \emph{\DUrole{n}{R\_ApEn}\DUrole{o}{=}\DUrole{default_value}{0.25}}}{}
Compute the approximate entropy, according to the paper
Utility of Approximate Entropy From Overnight Pulse Oximetry Data in the Diagnosis
of the Obstructive Sleep Apnea Syndrome
\begin{quote}\begin{description}
\item[{Parameters}] \leavevmode
\sphinxstyleliteralstrong{\sphinxupquote{signal}} \textendash{} 1\sphinxhyphen{}d array, of shape (N,) where N is the length of the signal

\item[{Returns}] \leavevmode
ApEn

\end{description}\end{quote}

\end{fulllineitems}

\index{dfa() (in module pobm.spo2.single\_biomarkers)@\spxentry{dfa()}\spxextra{in module pobm.spo2.single\_biomarkers}}

\begin{fulllineitems}
\phantomsection\label{\detokenize{pobm.spo2:pobm.spo2.single_biomarkers.dfa}}\pysiglinewithargsret{\sphinxcode{\sphinxupquote{pobm.spo2.single\_biomarkers.}}\sphinxbfcode{\sphinxupquote{dfa}}}{\emph{\DUrole{n}{signal}}, \emph{\DUrole{n}{DFA\_Window}\DUrole{o}{=}\DUrole{default_value}{20}}}{}
Compute DFA
\begin{quote}\begin{description}
\item[{Parameters}] \leavevmode\begin{itemize}
\item {} 
\sphinxstyleliteralstrong{\sphinxupquote{signal}} \textendash{} 1\sphinxhyphen{}d array, of shape (N,) where N is the length of the signal

\item {} 
\sphinxstyleliteralstrong{\sphinxupquote{DFA\_Window}} \textendash{} Length of window to calculate DFA biomarker.

\end{itemize}

\item[{Returns}] \leavevmode
DFA

\end{description}\end{quote}

\end{fulllineitems}

\index{lempel\_ziv() (in module pobm.spo2.single\_biomarkers)@\spxentry{lempel\_ziv()}\spxextra{in module pobm.spo2.single\_biomarkers}}

\begin{fulllineitems}
\phantomsection\label{\detokenize{pobm.spo2:pobm.spo2.single_biomarkers.lempel_ziv}}\pysiglinewithargsret{\sphinxcode{\sphinxupquote{pobm.spo2.single\_biomarkers.}}\sphinxbfcode{\sphinxupquote{lempel\_ziv}}}{\emph{\DUrole{n}{signal}}}{}
Compute lempel\sphinxhyphen{}ziv, according to the paper
Non\sphinxhyphen{}linear characteristics of blood oxygen saturation from nocturnal oximetry
for obstructive sleep apnoea detection
\begin{quote}\begin{description}
\item[{Parameters}] \leavevmode
\sphinxstyleliteralstrong{\sphinxupquote{signal}} \textendash{} 1\sphinxhyphen{}d array, of shape (N,) where N is the length of the signal

\item[{Returns}] \leavevmode
LZ

\end{description}\end{quote}

\end{fulllineitems}

\index{odi() (in module pobm.spo2.single\_biomarkers)@\spxentry{odi()}\spxextra{in module pobm.spo2.single\_biomarkers}}

\begin{fulllineitems}
\phantomsection\label{\detokenize{pobm.spo2:pobm.spo2.single_biomarkers.odi}}\pysiglinewithargsret{\sphinxcode{\sphinxupquote{pobm.spo2.single\_biomarkers.}}\sphinxbfcode{\sphinxupquote{odi}}}{\emph{\DUrole{n}{signal}}, \emph{\DUrole{n}{ODI\_Threshold}\DUrole{o}{=}\DUrole{default_value}{3}}}{}
Calculates the ODI from spo2 time series.
Suppose that the data has been preprocessed.
\begin{quote}\begin{description}
\item[{Parameters}] \leavevmode\begin{itemize}
\item {} 
\sphinxstyleliteralstrong{\sphinxupquote{signal}} \textendash{} The SpO2 signal, of shape (N,)

\item {} 
\sphinxstyleliteralstrong{\sphinxupquote{ODI\_Threshold}} \textendash{} Threshold to compute Oxygen Desaturation Index.

\end{itemize}

\item[{Returns}] \leavevmode
ODI

\end{description}\end{quote}

\end{fulllineitems}

\index{sampen() (in module pobm.spo2.single\_biomarkers)@\spxentry{sampen()}\spxextra{in module pobm.spo2.single\_biomarkers}}

\begin{fulllineitems}
\phantomsection\label{\detokenize{pobm.spo2:pobm.spo2.single_biomarkers.sampen}}\pysiglinewithargsret{\sphinxcode{\sphinxupquote{pobm.spo2.single\_biomarkers.}}\sphinxbfcode{\sphinxupquote{sampen}}}{\emph{\DUrole{n}{signal}}, \emph{\DUrole{n}{M\_Sampen}\DUrole{o}{=}\DUrole{default_value}{3}}, \emph{\DUrole{n}{R\_Sampen}\DUrole{o}{=}\DUrole{default_value}{0.2}}}{}
Compute the Sample Entropy
\begin{quote}\begin{description}
\item[{Parameters}] \leavevmode\begin{itemize}
\item {} 
\sphinxstyleliteralstrong{\sphinxupquote{signal}} \textendash{} 1\sphinxhyphen{}d array, of shape (N,) where N is the length of the signal

\item {} 
\sphinxstyleliteralstrong{\sphinxupquote{M\_Sampen}} \textendash{} Embedding dimension to compute SampEn.

\item {} 
\sphinxstyleliteralstrong{\sphinxupquote{R\_Sampen}} \textendash{} Tolerance to compute SampEn.

\end{itemize}

\item[{Returns}] \leavevmode
SampEn

\end{description}\end{quote}

\end{fulllineitems}



\subsubsection{Module contents}
\label{\detokenize{pobm.spo2:module-pobm.spo2}}\label{\detokenize{pobm.spo2:module-contents}}\index{module@\spxentry{module}!pobm.spo2@\spxentry{pobm.spo2}}\index{pobm.spo2@\spxentry{pobm.spo2}!module@\spxentry{module}}

\section{Submodules}
\label{\detokenize{pobm:submodules}}

\section{pobm.main}
\label{\detokenize{pobm:pobm-main}}

\section{pobm.prep}
\label{\detokenize{pobm:module-pobm.prep}}\label{\detokenize{pobm:pobm-prep}}\index{module@\spxentry{module}!pobm.prep@\spxentry{pobm.prep}}\index{pobm.prep@\spxentry{pobm.prep}!module@\spxentry{module}}\index{block\_data() (in module pobm.prep)@\spxentry{block\_data()}\spxextra{in module pobm.prep}}

\begin{fulllineitems}
\phantomsection\label{\detokenize{pobm:pobm.prep.block_data}}\pysiglinewithargsret{\sphinxcode{\sphinxupquote{pobm.prep.}}\sphinxbfcode{\sphinxupquote{block\_data}}}{\emph{\DUrole{n}{signal}}, \emph{\DUrole{n}{treshold}\DUrole{o}{=}\DUrole{default_value}{50}}}{}
Apply a block data filter to the SpO2 signal.
\begin{quote}\begin{description}
\item[{Parameters}] \leavevmode\begin{itemize}
\item {} 
\sphinxstyleliteralstrong{\sphinxupquote{signal}} \textendash{} 1\sphinxhyphen{}d array, of shape (N,) where N is the length of the signal

\item {} 
\sphinxstyleliteralstrong{\sphinxupquote{(}}\sphinxstyleliteralstrong{\sphinxupquote{Optional}}\sphinxstyleliteralstrong{\sphinxupquote{)}} (\sphinxstyleliteralemphasis{\sphinxupquote{treshold}}) \textendash{} treshold parameter for block data filter.

\end{itemize}

\item[{Returns}] \leavevmode
preprocessed signal, 1\sphinxhyphen{}d numpy array.

\end{description}\end{quote}

\end{fulllineitems}

\index{dfilter() (in module pobm.prep)@\spxentry{dfilter()}\spxextra{in module pobm.prep}}

\begin{fulllineitems}
\phantomsection\label{\detokenize{pobm:pobm.prep.dfilter}}\pysiglinewithargsret{\sphinxcode{\sphinxupquote{pobm.prep.}}\sphinxbfcode{\sphinxupquote{dfilter}}}{\emph{\DUrole{n}{signal}}, \emph{\DUrole{n}{Diff}\DUrole{o}{=}\DUrole{default_value}{4}}}{}
Apply Delta Filter to the signal.
\begin{quote}\begin{description}
\item[{Parameters}] \leavevmode\begin{itemize}
\item {} 
\sphinxstyleliteralstrong{\sphinxupquote{signal}} \textendash{} 1\sphinxhyphen{}d array, of shape (N,) where N is the length of the signal

\item {} 
\sphinxstyleliteralstrong{\sphinxupquote{Diff}} \textendash{} parameter of the delta filter.

\end{itemize}

\item[{Returns}] \leavevmode
preprocessed signal, 1\sphinxhyphen{}d numpy array.

\end{description}\end{quote}

\end{fulllineitems}

\index{median\_spo2() (in module pobm.prep)@\spxentry{median\_spo2()}\spxextra{in module pobm.prep}}

\begin{fulllineitems}
\phantomsection\label{\detokenize{pobm:pobm.prep.median_spo2}}\pysiglinewithargsret{\sphinxcode{\sphinxupquote{pobm.prep.}}\sphinxbfcode{\sphinxupquote{median\_spo2}}}{\emph{\DUrole{n}{signal\_spo2}}, \emph{\DUrole{n}{FilterLength}\DUrole{o}{=}\DUrole{default_value}{9}}}{}
Apply a median filter to the SpO2 signal.
Median filter used to smooth the spo2 time series and avoid sporadic increase/decrease of spo2 which could 
affect the detection of the desaturations.
Assumption: any missing/abnormal values are represented as ‘np.nan’
\begin{quote}\begin{description}
\item[{Parameters}] \leavevmode\begin{itemize}
\item {} 
\sphinxstyleliteralstrong{\sphinxupquote{signal}} \textendash{} 1\sphinxhyphen{}d array, of shape (N,) where N is the length of the signal

\item {} 
\sphinxstyleliteralstrong{\sphinxupquote{(}}\sphinxstyleliteralstrong{\sphinxupquote{Optional}}\sphinxstyleliteralstrong{\sphinxupquote{)}} (\sphinxstyleliteralemphasis{\sphinxupquote{FilterLength}}) \textendash{} The length of the filter.

\end{itemize}

\item[{Returns}] \leavevmode
preprocessed signal, 1\sphinxhyphen{}d numpy array.

\end{description}\end{quote}

\end{fulllineitems}

\index{resamp\_spo2() (in module pobm.prep)@\spxentry{resamp\_spo2()}\spxextra{in module pobm.prep}}

\begin{fulllineitems}
\phantomsection\label{\detokenize{pobm:pobm.prep.resamp_spo2}}\pysiglinewithargsret{\sphinxcode{\sphinxupquote{pobm.prep.}}\sphinxbfcode{\sphinxupquote{resamp\_spo2}}}{\emph{\DUrole{n}{signal}}, \emph{\DUrole{n}{OriginalFreq}}}{}
Resample the SpO2 signal to 1Hz.
Assumption: any missing/abnormal values are represented as ‘np.nan’
\begin{quote}\begin{description}
\item[{Parameters}] \leavevmode\begin{itemize}
\item {} 
\sphinxstyleliteralstrong{\sphinxupquote{signal}} \textendash{} 1\sphinxhyphen{}d array, of shape (N,) where N is the length of the signal

\item {} 
\sphinxstyleliteralstrong{\sphinxupquote{OriginalFreq}} \textendash{} the original frequency.

\end{itemize}

\item[{Returns}] \leavevmode
resampled signal, 1\sphinxhyphen{}d numpy array, the resampled spo2 time series at 1Hz

\end{description}\end{quote}

\end{fulllineitems}

\index{set\_range() (in module pobm.prep)@\spxentry{set\_range()}\spxextra{in module pobm.prep}}

\begin{fulllineitems}
\phantomsection\label{\detokenize{pobm:pobm.prep.set_range}}\pysiglinewithargsret{\sphinxcode{\sphinxupquote{pobm.prep.}}\sphinxbfcode{\sphinxupquote{set\_range}}}{\emph{\DUrole{n}{signal}}, \emph{\DUrole{n}{Range\_min}\DUrole{o}{=}\DUrole{default_value}{50}}, \emph{\DUrole{n}{Range\_max}\DUrole{o}{=}\DUrole{default_value}{100}}}{}
Range function. Remove values lower than 50 or greater than 100, considered as non\sphinxhyphen{}physiological
\begin{quote}\begin{description}
\item[{Parameters}] \leavevmode
\sphinxstyleliteralstrong{\sphinxupquote{signal}} \textendash{} 1\sphinxhyphen{}d array, of shape (N,) where N is the length of the signal

\item[{Returns}] \leavevmode
preprocessed signal, 1\sphinxhyphen{}d numpy array.

\end{description}\end{quote}

\end{fulllineitems}



\section{Module contents}
\label{\detokenize{pobm:module-pobm}}\label{\detokenize{pobm:module-contents}}\index{module@\spxentry{module}!pobm@\spxentry{pobm}}\index{pobm@\spxentry{pobm}!module@\spxentry{module}}

\chapter{Indices and tables}
\label{\detokenize{index:indices-and-tables}}\begin{itemize}
\item {} 
\DUrole{xref,std,std-ref}{genindex}

\item {} 
\DUrole{xref,std,std-ref}{modindex}

\item {} 
\DUrole{xref,std,std-ref}{search}

\end{itemize}


\renewcommand{\indexname}{Python Module Index}
\begin{sphinxtheindex}
\let\bigletter\sphinxstyleindexlettergroup
\bigletter{p}
\item\relax\sphinxstyleindexentry{pobm}\sphinxstyleindexpageref{pobm:\detokenize{module-pobm}}
\item\relax\sphinxstyleindexentry{pobm.obm}\sphinxstyleindexpageref{pobm.obm:\detokenize{module-pobm.obm}}
\item\relax\sphinxstyleindexentry{pobm.obm.burden}\sphinxstyleindexpageref{pobm.obm:\detokenize{module-pobm.obm.burden}}
\item\relax\sphinxstyleindexentry{pobm.obm.complex}\sphinxstyleindexpageref{pobm.obm:\detokenize{module-pobm.obm.complex}}
\item\relax\sphinxstyleindexentry{pobm.obm.desat}\sphinxstyleindexpageref{pobm.obm:\detokenize{module-pobm.obm.desat}}
\item\relax\sphinxstyleindexentry{pobm.obm.general}\sphinxstyleindexpageref{pobm.obm:\detokenize{module-pobm.obm.general}}
\item\relax\sphinxstyleindexentry{pobm.obm.periodicity}\sphinxstyleindexpageref{pobm.obm:\detokenize{module-pobm.obm.periodicity}}
\item\relax\sphinxstyleindexentry{pobm.prep}\sphinxstyleindexpageref{pobm:\detokenize{module-pobm.prep}}
\item\relax\sphinxstyleindexentry{pobm.spo2}\sphinxstyleindexpageref{pobm.spo2:\detokenize{module-pobm.spo2}}
\item\relax\sphinxstyleindexentry{pobm.spo2.single\_biomarkers}\sphinxstyleindexpageref{pobm.spo2:\detokenize{module-pobm.spo2.single_biomarkers}}
\end{sphinxtheindex}

\renewcommand{\indexname}{Index}
\printindex
\end{document}