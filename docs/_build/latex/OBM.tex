%% Generated by Sphinx.
\def\sphinxdocclass{report}
\documentclass[letterpaper,10pt,english]{sphinxmanual}
\ifdefined\pdfpxdimen
   \let\sphinxpxdimen\pdfpxdimen\else\newdimen\sphinxpxdimen
\fi \sphinxpxdimen=.75bp\relax

\PassOptionsToPackage{warn}{textcomp}
\usepackage[utf8]{inputenc}
\ifdefined\DeclareUnicodeCharacter
% support both utf8 and utf8x syntaxes
\edef\sphinxdqmaybe{\ifdefined\DeclareUnicodeCharacterAsOptional\string"\fi}
  \DeclareUnicodeCharacter{\sphinxdqmaybe00A0}{\nobreakspace}
  \DeclareUnicodeCharacter{\sphinxdqmaybe2500}{\sphinxunichar{2500}}
  \DeclareUnicodeCharacter{\sphinxdqmaybe2502}{\sphinxunichar{2502}}
  \DeclareUnicodeCharacter{\sphinxdqmaybe2514}{\sphinxunichar{2514}}
  \DeclareUnicodeCharacter{\sphinxdqmaybe251C}{\sphinxunichar{251C}}
  \DeclareUnicodeCharacter{\sphinxdqmaybe2572}{\textbackslash}
\fi
\usepackage{cmap}
\usepackage[T1]{fontenc}
\usepackage{amsmath,amssymb,amstext}
\usepackage{babel}
\usepackage{times}
\usepackage[Sonny]{fncychap}
\ChNameVar{\Large\normalfont\sffamily}
\ChTitleVar{\Large\normalfont\sffamily}
\usepackage{sphinx}

\fvset{fontsize=\small}
\usepackage{geometry}

% Include hyperref last.
\usepackage{hyperref}
% Fix anchor placement for figures with captions.
\usepackage{hypcap}% it must be loaded after hyperref.
% Set up styles of URL: it should be placed after hyperref.
\urlstyle{same}
\addto\captionsenglish{\renewcommand{\contentsname}{Contents:}}

\addto\captionsenglish{\renewcommand{\figurename}{Fig.\@ }}
\makeatletter
\def\fnum@figure{\figurename\thefigure{}}
\makeatother
\addto\captionsenglish{\renewcommand{\tablename}{Table }}
\makeatletter
\def\fnum@table{\tablename\thetable{}}
\makeatother
\addto\captionsenglish{\renewcommand{\literalblockname}{Listing}}

\addto\captionsenglish{\renewcommand{\literalblockcontinuedname}{continued from previous page}}
\addto\captionsenglish{\renewcommand{\literalblockcontinuesname}{continues on next page}}
\addto\captionsenglish{\renewcommand{\sphinxnonalphabeticalgroupname}{Non-alphabetical}}
\addto\captionsenglish{\renewcommand{\sphinxsymbolsname}{Symbols}}
\addto\captionsenglish{\renewcommand{\sphinxnumbersname}{Numbers}}

\addto\extrasenglish{\def\pageautorefname{page}}

\setcounter{tocdepth}{3}
\setcounter{secnumdepth}{3}


\title{OBM Documentation}
\date{Sep 24, 2020}
\release{}
\author{AIMLab}
\newcommand{\sphinxlogo}{\vbox{}}
\renewcommand{\releasename}{}
\makeindex
\begin{document}

\ifdefined\shorthandoff
  \ifnum\catcode`\=\string=\active\shorthandoff{=}\fi
  \ifnum\catcode`\"=\active\shorthandoff{"}\fi
\fi

\pagestyle{empty}
\sphinxmaketitle
\pagestyle{plain}
\sphinxtableofcontents
\pagestyle{normal}
\phantomsection\label{\detokenize{index::doc}}



\chapter{OBM package}
\label{\detokenize{OBM:module-OBM}}\label{\detokenize{OBM:obm-package}}\label{\detokenize{OBM::doc}}\index{OBM (module)@\spxentry{OBM}\spxextra{module}}

\section{OBM.ComplexityMeasures class}
\label{\detokenize{OBM:module-OBM.ComplexityMeasures}}\label{\detokenize{OBM:obm-complexitymeasures-class}}\index{OBM.ComplexityMeasures (module)@\spxentry{OBM.ComplexityMeasures}\spxextra{module}}\index{ComplexityMeasures (class in OBM.ComplexityMeasures)@\spxentry{ComplexityMeasures}\spxextra{class in OBM.ComplexityMeasures}}

\begin{fulllineitems}
\phantomsection\label{\detokenize{OBM:OBM.ComplexityMeasures.ComplexityMeasures}}\pysiglinewithargsret{\sphinxbfcode{\sphinxupquote{class }}\sphinxcode{\sphinxupquote{OBM.ComplexityMeasures.}}\sphinxbfcode{\sphinxupquote{ComplexityMeasures}}}{\emph{CTM\_Threshold=0.25}, \emph{DFA\_Window=20}, \emph{M\_Sampen=3}, \emph{R\_Sampen=0.2}}{}
Bases: \sphinxcode{\sphinxupquote{object}}

Class that calculates Complexity Features from spo2 time series.
Suppose that the data has been preprocessed.
\begin{description}
\item[{:param}] \leavevmode
signal: 1-d array, of shape (N,) where N is the length of the signal
CTM\_Threshold: Radius of Central Tendency Measure.
DFA\_Window: Length of window to calculate DFA biomarker.
M\_Sampen: Embedding dimension to compute SampEn.
R\_Sampen: Tolerance to compute SampEn.

\end{description}
\index{compute() (OBM.ComplexityMeasures.ComplexityMeasures method)@\spxentry{compute()}\spxextra{OBM.ComplexityMeasures.ComplexityMeasures method}}

\begin{fulllineitems}
\phantomsection\label{\detokenize{OBM:OBM.ComplexityMeasures.ComplexityMeasures.compute}}\pysiglinewithargsret{\sphinxbfcode{\sphinxupquote{compute}}}{\emph{signal}}{{ $\rightarrow$ OBM.\_ResultsClasses.ComplexityMeasuresResults}}~\begin{quote}\begin{description}
\item[{Parameters}] \leavevmode
\sphinxstyleliteralstrong{\sphinxupquote{signal}} \textendash{} 1-d array, of shape (N,) where N is the length of the signal

\item[{Returns}] \leavevmode
\begin{description}
\item[{ComplexityMeasuresResults class containing the following features:}] \leavevmode\begin{itemize}
\item {} 
ApEn: Approximate Entropy.

\item {} 
LZ: Lempel-Ziv complexity.

\item {} 
CTM: Central Tendency Measure.

\item {} 
SampEn: Sample Entropy.

\item {} 
DFA: Detrended Fluctuation Analysis.

\end{itemize}

\end{description}


\end{description}\end{quote}

\end{fulllineitems}


\end{fulllineitems}



\section{OBM.DesaturationsMeasures class}
\label{\detokenize{OBM:module-OBM.DesaturationsMeasures}}\label{\detokenize{OBM:obm-desaturationsmeasures-class}}\index{OBM.DesaturationsMeasures (module)@\spxentry{OBM.DesaturationsMeasures}\spxextra{module}}\index{DesaturationsMeasures (class in OBM.DesaturationsMeasures)@\spxentry{DesaturationsMeasures}\spxextra{class in OBM.DesaturationsMeasures}}

\begin{fulllineitems}
\phantomsection\label{\detokenize{OBM:OBM.DesaturationsMeasures.DesaturationsMeasures}}\pysiglinewithargsret{\sphinxbfcode{\sphinxupquote{class }}\sphinxcode{\sphinxupquote{OBM.DesaturationsMeasures.}}\sphinxbfcode{\sphinxupquote{DesaturationsMeasures}}}{\emph{begin}, \emph{end}}{}
Bases: \sphinxcode{\sphinxupquote{object}}

Class that calculates the Desaturation Features from spo2 time series.
Suppose that the data has been preprocessed.
\begin{quote}\begin{description}
\item[{Parameters}] \leavevmode\begin{itemize}
\item {} 
\sphinxstyleliteralstrong{\sphinxupquote{begin}} \textendash{} List of indices of beginning of each desaturation event.

\item {} 
\sphinxstyleliteralstrong{\sphinxupquote{end}} \textendash{} List of indices of end of each desaturation event.

\end{itemize}

\end{description}\end{quote}
\index{compute() (OBM.DesaturationsMeasures.DesaturationsMeasures method)@\spxentry{compute()}\spxextra{OBM.DesaturationsMeasures.DesaturationsMeasures method}}

\begin{fulllineitems}
\phantomsection\label{\detokenize{OBM:OBM.DesaturationsMeasures.DesaturationsMeasures.compute}}\pysiglinewithargsret{\sphinxbfcode{\sphinxupquote{compute}}}{\emph{signal}}{{ $\rightarrow$ OBM.\_ResultsClasses.DesaturationsMeasuresResults}}~\begin{quote}\begin{description}
\item[{Parameters}] \leavevmode
\sphinxstyleliteralstrong{\sphinxupquote{signal}} \textendash{} 1-d array, of shape (N,) where N is the length of the signal

\item[{Returns}] \leavevmode
\begin{description}
\item[{DesaturationsMeasuresResults class containing the following features:}] \leavevmode\begin{itemize}
\item {} 
DL\_u: Mean of desaturation length

\item {} 
DL\_sd: Standard deviation of desaturation length

\item {} 
DA100\_u: Mean of desaturation area using 100\% as baseline.

\item {} 
DA100\_sd: Standard deviation of desaturation area using 100\% as baseline

\item {} 
DAmax\_u: Mean of desaturation area using max value as baseline.

\item {} 
DAmax\_sd: Standard deviation of desaturation area using max value as baseline

\item {} 
DD100\_u: Mean of depth desaturation from 100\%.

\item {} 
DD100\_sd: Standard deviation of depth desaturation from 100\%.

\item {} 
DDmax\_u: Mean of depth desaturation from max value.

\item {} 
DDmax\_sd: Standard deviation of depth desaturation from max value.

\item {} 
DS\_u: Mean of the desaturation slope.

\item {} 
DS\_sd: Standard deviation of the desaturation slope.

\item {} 
TD\_u: Mean of time between two consecutive desaturation events.

\item {} 
TD\_sd: Standard deviation of time between 2 consecutive desaturation events.

\end{itemize}

\end{description}


\end{description}\end{quote}

\end{fulllineitems}

\index{desat\_embedding() (OBM.DesaturationsMeasures.DesaturationsMeasures method)@\spxentry{desat\_embedding()}\spxextra{OBM.DesaturationsMeasures.DesaturationsMeasures method}}

\begin{fulllineitems}
\phantomsection\label{\detokenize{OBM:OBM.DesaturationsMeasures.DesaturationsMeasures.desat_embedding}}\pysiglinewithargsret{\sphinxbfcode{\sphinxupquote{desat\_embedding}}}{}{}
Help function for the class
\begin{quote}\begin{description}
\item[{Returns}] \leavevmode
helper arrays containing the information about desaturation lengths and areas.

\end{description}\end{quote}

\end{fulllineitems}


\end{fulllineitems}



\section{OBM.HypoxicBurdenMeasures class}
\label{\detokenize{OBM:module-OBM.HypoxicBurdenMeasures}}\label{\detokenize{OBM:obm-hypoxicburdenmeasures-class}}\index{OBM.HypoxicBurdenMeasures (module)@\spxentry{OBM.HypoxicBurdenMeasures}\spxextra{module}}\index{HypoxicBurdenMeasures (class in OBM.HypoxicBurdenMeasures)@\spxentry{HypoxicBurdenMeasures}\spxextra{class in OBM.HypoxicBurdenMeasures}}

\begin{fulllineitems}
\phantomsection\label{\detokenize{OBM:OBM.HypoxicBurdenMeasures.HypoxicBurdenMeasures}}\pysiglinewithargsret{\sphinxbfcode{\sphinxupquote{class }}\sphinxcode{\sphinxupquote{OBM.HypoxicBurdenMeasures.}}\sphinxbfcode{\sphinxupquote{HypoxicBurdenMeasures}}}{\emph{begin}, \emph{end}, \emph{CT\_Threshold=90}, \emph{CA\_Baseline=None}}{}
Bases: \sphinxcode{\sphinxupquote{object}}

Class that calculates Hypoxic Burden Features from spo2 time series.
Suppose that the data has been preprocessed.
\begin{quote}\begin{description}
\item[{Parameters}] \leavevmode\begin{itemize}
\item {} 
\sphinxstyleliteralstrong{\sphinxupquote{begin}} \textendash{} List of indices of beginning of each desaturation event.

\item {} 
\sphinxstyleliteralstrong{\sphinxupquote{end}} \textendash{} List of indices of end of each desaturation event.

\item {} 
\sphinxstyleliteralstrong{\sphinxupquote{CT\_Threshold}} \textendash{} Percentage of the time spent below the “CT\_Threshold” \% oxygen saturation level.

\item {} 
\sphinxstyleliteralstrong{\sphinxupquote{CA\_Baseline}} \textendash{} Baseline to compute the CA feature. Default value is mean of the signal.

\end{itemize}

\end{description}\end{quote}
\index{compute() (OBM.HypoxicBurdenMeasures.HypoxicBurdenMeasures method)@\spxentry{compute()}\spxextra{OBM.HypoxicBurdenMeasures.HypoxicBurdenMeasures method}}

\begin{fulllineitems}
\phantomsection\label{\detokenize{OBM:OBM.HypoxicBurdenMeasures.HypoxicBurdenMeasures.compute}}\pysiglinewithargsret{\sphinxbfcode{\sphinxupquote{compute}}}{\emph{signal}}{}~\begin{quote}\begin{description}
\item[{Parameters}] \leavevmode
\sphinxstyleliteralstrong{\sphinxupquote{signal}} \textendash{} 1-d array, of shape (N,) where N is the length of the signal

\item[{Returns}] \leavevmode
\begin{description}
\item[{HypoxicBurdenMeasuresResults class containing the following features:}] \leavevmode\begin{itemize}
\item {} 
CA: Integral SpO2 below the xx SpO2 level normalized by the total recording time

\item {} 
CT: Percentage of the time spent below the xx\% oxygen saturation level

\item {} 
POD: Percentage of oxygen desaturation events

\item {} 
AODmax: The area under the oxygen desaturation event curve, using the maximum SpO2 value as baseline
and normalized by the total recording time

\item {} 
AOD100: Cumulative area of desaturations under the 100\% SpO2 level as baseline and normalized
by the total recording time

\end{itemize}

\end{description}


\end{description}\end{quote}

\end{fulllineitems}


\end{fulllineitems}



\section{OBM.ODIMeasure class}
\label{\detokenize{OBM:module-OBM.ODIMeasure}}\label{\detokenize{OBM:obm-odimeasure-class}}\index{OBM.ODIMeasure (module)@\spxentry{OBM.ODIMeasure}\spxextra{module}}\index{ODIMeasure (class in OBM.ODIMeasure)@\spxentry{ODIMeasure}\spxextra{class in OBM.ODIMeasure}}

\begin{fulllineitems}
\phantomsection\label{\detokenize{OBM:OBM.ODIMeasure.ODIMeasure}}\pysiglinewithargsret{\sphinxbfcode{\sphinxupquote{class }}\sphinxcode{\sphinxupquote{OBM.ODIMeasure.}}\sphinxbfcode{\sphinxupquote{ODIMeasure}}}{\emph{ODI\_Threshold=3}}{}
Bases: \sphinxcode{\sphinxupquote{object}}

Class that calculates the ODI from spo2 time series.
Suppose that the data has been preprocessed.
\begin{quote}\begin{description}
\item[{Parameters}] \leavevmode
\sphinxstyleliteralstrong{\sphinxupquote{ODI\_Threshold}} \textendash{} Threshold to compute Oxygen Desaturation Index.

\end{description}\end{quote}
\index{compute() (OBM.ODIMeasure.ODIMeasure method)@\spxentry{compute()}\spxextra{OBM.ODIMeasure.ODIMeasure method}}

\begin{fulllineitems}
\phantomsection\label{\detokenize{OBM:OBM.ODIMeasure.ODIMeasure.compute}}\pysiglinewithargsret{\sphinxbfcode{\sphinxupquote{compute}}}{\emph{signal}}{{ $\rightarrow$ OBM.\_ResultsClasses.ODIMeasureResult}}~\begin{quote}\begin{description}
\item[{Parameters}] \leavevmode
\sphinxstyleliteralstrong{\sphinxupquote{signal}} \textendash{} The SpO2 signal, of shape (N,)

\item[{Returns}] \leavevmode
\begin{description}
\item[{ODIMeasureResult class containing the following features:}] \leavevmode\begin{itemize}
\item {} 
ODI: the average number of desaturation events per hour.

\item {} 
begin: List of indices of beginning of each desaturation event.

\item {} 
end: List of indices of end of each desaturation event.

\end{itemize}

\end{description}


\end{description}\end{quote}

\end{fulllineitems}


\end{fulllineitems}



\section{OBM.OverallGeneralMeasures class}
\label{\detokenize{OBM:module-OBM.OverallGeneralMeasures}}\label{\detokenize{OBM:obm-overallgeneralmeasures-class}}\index{OBM.OverallGeneralMeasures (module)@\spxentry{OBM.OverallGeneralMeasures}\spxextra{module}}\index{OverallGeneralMeasures (class in OBM.OverallGeneralMeasures)@\spxentry{OverallGeneralMeasures}\spxextra{class in OBM.OverallGeneralMeasures}}

\begin{fulllineitems}
\phantomsection\label{\detokenize{OBM:OBM.OverallGeneralMeasures.OverallGeneralMeasures}}\pysiglinewithargsret{\sphinxbfcode{\sphinxupquote{class }}\sphinxcode{\sphinxupquote{OBM.OverallGeneralMeasures.}}\sphinxbfcode{\sphinxupquote{OverallGeneralMeasures}}}{\emph{ZC\_Baseline=None}, \emph{percentile=1}, \emph{M\_Threshold=2}, \emph{DI\_Window=12}}{}
Bases: \sphinxcode{\sphinxupquote{object}}

Class that calculates Overall General Features from spo2 time series.
Suppose that the data has been preprocessed.
\begin{quote}\begin{description}
\item[{Parameters}] \leavevmode\begin{itemize}
\item {} 
\sphinxstyleliteralstrong{\sphinxupquote{ZC\_Baseline}} \textendash{} Baseline for calculating number of zero-crossing points.

\item {} 
\sphinxstyleliteralstrong{\sphinxupquote{percentile}} \textendash{} Percentile to perform. For example, for percentile 1, the argument should be 1

\item {} 
\sphinxstyleliteralstrong{\sphinxupquote{M\_Threshold}} \textendash{} Percentage of the signal M\_Threshold \% below median oxygen saturation. Typically use 1,2 or 5

\end{itemize}

\end{description}\end{quote}
\index{compute() (OBM.OverallGeneralMeasures.OverallGeneralMeasures method)@\spxentry{compute()}\spxextra{OBM.OverallGeneralMeasures.OverallGeneralMeasures method}}

\begin{fulllineitems}
\phantomsection\label{\detokenize{OBM:OBM.OverallGeneralMeasures.OverallGeneralMeasures.compute}}\pysiglinewithargsret{\sphinxbfcode{\sphinxupquote{compute}}}{\emph{signal}}{{ $\rightarrow$ OBM.\_ResultsClasses.OverallGeneralMeasuresResult}}~\begin{quote}\begin{description}
\item[{Parameters}] \leavevmode
\sphinxstyleliteralstrong{\sphinxupquote{signal}} \textendash{} 1-d array, of shape (N,) where N is the length of the signal

\item[{Returns}] \leavevmode
\begin{description}
\item[{OveralGeneralMeasuresResult class containing the following features:}] \leavevmode\begin{itemize}
\item {} 
AV: Average of the signal.

\item {} 
MED: Median of the signal.

\item {} 
Min: Minimum value of the signal.

\item {} 
SD: Std of the signal.

\item {} 
RG: SpO2 range (difference between the max and min value).

\item {} 
P: percentile.

\item {} 
M: Percentage of the signal x\% below median oxygen saturation.

\item {} 
ZC: Number of zero-crossing points.

\item {} 
DI: Delta Index.

\end{itemize}

\end{description}


\end{description}\end{quote}

\end{fulllineitems}


\end{fulllineitems}



\section{OBM.PeriodicityMeasures class}
\label{\detokenize{OBM:module-OBM.PeriodicityMeasures}}\label{\detokenize{OBM:obm-periodicitymeasures-class}}\index{OBM.PeriodicityMeasures (module)@\spxentry{OBM.PeriodicityMeasures}\spxextra{module}}\index{PRSAMeasures (class in OBM.PeriodicityMeasures)@\spxentry{PRSAMeasures}\spxextra{class in OBM.PeriodicityMeasures}}

\begin{fulllineitems}
\phantomsection\label{\detokenize{OBM:OBM.PeriodicityMeasures.PRSAMeasures}}\pysiglinewithargsret{\sphinxbfcode{\sphinxupquote{class }}\sphinxcode{\sphinxupquote{OBM.PeriodicityMeasures.}}\sphinxbfcode{\sphinxupquote{PRSAMeasures}}}{\emph{PRSA\_Window=10}, \emph{K\_AC=2}}{}
Bases: \sphinxcode{\sphinxupquote{object}}

Function that calculates PRSA Features from spo2 time series.
Suppose that the data has been preprocessed.
\begin{description}
\item[{:param}] \leavevmode
PRSA\_Window: Fragment duration of PRSA.
K\_AC: Number of values to shift when computing autocorrelation

\end{description}
\index{compute() (OBM.PeriodicityMeasures.PRSAMeasures method)@\spxentry{compute()}\spxextra{OBM.PeriodicityMeasures.PRSAMeasures method}}

\begin{fulllineitems}
\phantomsection\label{\detokenize{OBM:OBM.PeriodicityMeasures.PRSAMeasures.compute}}\pysiglinewithargsret{\sphinxbfcode{\sphinxupquote{compute}}}{\emph{signal}}{{ $\rightarrow$ OBM.\_ResultsClasses.PRSAResults}}~\begin{quote}\begin{description}
\item[{Parameters}] \leavevmode
\sphinxstyleliteralstrong{\sphinxupquote{signal}} \textendash{} 1-d array, of shape (N,) where N is the length of the signal

\item[{Returns}] \leavevmode
\begin{description}
\item[{PRSAResults class containing the following features:}] \leavevmode\begin{itemize}
\item {} 
PRSAc: PRSA capacity.

\item {} 
PRSAad: PRSA amplitude difference.

\item {} 
PRSAos: PRSA overall slope.

\item {} 
PRSAsb: PRSA slope before the anchor point.

\item {} 
PRSAsa: PRSA slope after the anchor point.

\item {} 
AC: Autocorrelation.

\end{itemize}

\end{description}


\end{description}\end{quote}

\end{fulllineitems}


\end{fulllineitems}

\index{PSDMeasures (class in OBM.PeriodicityMeasures)@\spxentry{PSDMeasures}\spxextra{class in OBM.PeriodicityMeasures}}

\begin{fulllineitems}
\phantomsection\label{\detokenize{OBM:OBM.PeriodicityMeasures.PSDMeasures}}\pysigline{\sphinxbfcode{\sphinxupquote{class }}\sphinxcode{\sphinxupquote{OBM.PeriodicityMeasures.}}\sphinxbfcode{\sphinxupquote{PSDMeasures}}}
Bases: \sphinxcode{\sphinxupquote{object}}

Function that calculates PSD Features from spo2 time series.
Suppose that the data has been preprocessed.
\index{compute() (OBM.PeriodicityMeasures.PSDMeasures method)@\spxentry{compute()}\spxextra{OBM.PeriodicityMeasures.PSDMeasures method}}

\begin{fulllineitems}
\phantomsection\label{\detokenize{OBM:OBM.PeriodicityMeasures.PSDMeasures.compute}}\pysiglinewithargsret{\sphinxbfcode{\sphinxupquote{compute}}}{\emph{signal}}{{ $\rightarrow$ OBM.\_ResultsClasses.PSDResults}}~\begin{description}
\item[{:param}] \leavevmode
signal: The SpO2 signal, of shape (N,)

\end{description}
\begin{quote}\begin{description}
\item[{Returns}] \leavevmode
\begin{description}
\item[{PSDResults class containing the following features:}] \leavevmode\begin{itemize}
\item {} 
PSD\_total: The amplitude of the spectral signal.

\item {} 
PSD\_band: The amplitude of the signal multiplied by a band-pass filter between 0.014 and 0.033 Hz.

\item {} 
PSD\_ratio: The ratio between PSD\_total and PSD\_band.

\item {} 
PDS\_peak: The max value of the FFT into the band 0.014-0.033 Hz.

\end{itemize}

\end{description}


\end{description}\end{quote}

\end{fulllineitems}


\end{fulllineitems}



\section{OBM.Preprocessing class}
\label{\detokenize{OBM:module-OBM.Preprocessing}}\label{\detokenize{OBM:obm-preprocessing-class}}\index{OBM.Preprocessing (module)@\spxentry{OBM.Preprocessing}\spxextra{module}}\index{block\_data() (in module OBM.Preprocessing)@\spxentry{block\_data()}\spxextra{in module OBM.Preprocessing}}

\begin{fulllineitems}
\phantomsection\label{\detokenize{OBM:OBM.Preprocessing.block_data}}\pysiglinewithargsret{\sphinxcode{\sphinxupquote{OBM.Preprocessing.}}\sphinxbfcode{\sphinxupquote{block\_data}}}{\emph{signal}, \emph{treshold=50}}{}
Apply a block data filter to the SpO2 signal.
\begin{quote}\begin{description}
\item[{Parameters}] \leavevmode\begin{itemize}
\item {} 
\sphinxstyleliteralstrong{\sphinxupquote{signal}} \textendash{} 1-d array, of shape (N,) where N is the length of the signal

\item {} 
\sphinxstyleliteralstrong{\sphinxupquote{(}}\sphinxstyleliteralstrong{\sphinxupquote{Optional}}\sphinxstyleliteralstrong{\sphinxupquote{)}} (\sphinxstyleliteralemphasis{\sphinxupquote{treshold}}) \textendash{} treshold parameter for block data filter.

\end{itemize}

\item[{Returns}] \leavevmode
preprocessed signal, 1-d numpy array.

\end{description}\end{quote}

\end{fulllineitems}

\index{dlta\_filter() (in module OBM.Preprocessing)@\spxentry{dlta\_filter()}\spxextra{in module OBM.Preprocessing}}

\begin{fulllineitems}
\phantomsection\label{\detokenize{OBM:OBM.Preprocessing.dlta_filter}}\pysiglinewithargsret{\sphinxcode{\sphinxupquote{OBM.Preprocessing.}}\sphinxbfcode{\sphinxupquote{dlta\_filter}}}{\emph{signal}, \emph{Diff=4}}{}
Apply Delta Filter to the signal.
\begin{quote}\begin{description}
\item[{Parameters}] \leavevmode\begin{itemize}
\item {} 
\sphinxstyleliteralstrong{\sphinxupquote{signal}} \textendash{} 1-d array, of shape (N,) where N is the length of the signal

\item {} 
\sphinxstyleliteralstrong{\sphinxupquote{Diff}} \textendash{} parameter of the delta filter.

\end{itemize}

\item[{Returns}] \leavevmode
preprocessed signal, 1-d numpy array.

\end{description}\end{quote}

\end{fulllineitems}

\index{median\_spo2() (in module OBM.Preprocessing)@\spxentry{median\_spo2()}\spxextra{in module OBM.Preprocessing}}

\begin{fulllineitems}
\phantomsection\label{\detokenize{OBM:OBM.Preprocessing.median_spo2}}\pysiglinewithargsret{\sphinxcode{\sphinxupquote{OBM.Preprocessing.}}\sphinxbfcode{\sphinxupquote{median\_spo2}}}{\emph{signal\_spo2}, \emph{FilterLength=9}}{}
Apply a median filter to the SpO2 signal.
Median filter used to smooth the spo2 time series and avoid sporadic increase/decrease of spo2 which could 
affect the detection of the desaturations.
Assumption: any missing/abnormal values are represented as ‘np.nan’
\begin{quote}\begin{description}
\item[{Parameters}] \leavevmode\begin{itemize}
\item {} 
\sphinxstyleliteralstrong{\sphinxupquote{signal}} \textendash{} 1-d array, of shape (N,) where N is the length of the signal

\item {} 
\sphinxstyleliteralstrong{\sphinxupquote{(}}\sphinxstyleliteralstrong{\sphinxupquote{Optional}}\sphinxstyleliteralstrong{\sphinxupquote{)}} (\sphinxstyleliteralemphasis{\sphinxupquote{FilterLength}}) \textendash{} The length of the filter.

\end{itemize}

\item[{Returns}] \leavevmode
preprocessed signal, 1-d numpy array.

\end{description}\end{quote}

\end{fulllineitems}

\index{resamp\_spo2() (in module OBM.Preprocessing)@\spxentry{resamp\_spo2()}\spxextra{in module OBM.Preprocessing}}

\begin{fulllineitems}
\phantomsection\label{\detokenize{OBM:OBM.Preprocessing.resamp_spo2}}\pysiglinewithargsret{\sphinxcode{\sphinxupquote{OBM.Preprocessing.}}\sphinxbfcode{\sphinxupquote{resamp\_spo2}}}{\emph{signal}, \emph{OriginalFreq}}{}
Resample the SpO2 signal to 1Hz.
Assumption: any missing/abnormal values are represented as ‘np.nan’
\begin{quote}\begin{description}
\item[{Parameters}] \leavevmode\begin{itemize}
\item {} 
\sphinxstyleliteralstrong{\sphinxupquote{signal}} \textendash{} 1-d array, of shape (N,) where N is the length of the signal

\item {} 
\sphinxstyleliteralstrong{\sphinxupquote{OriginalFreq}} \textendash{} the original frequency.

\end{itemize}

\item[{Returns}] \leavevmode
resampled signal, 1-d numpy array, the resampled spo2 time series at 1Hz

\end{description}\end{quote}

\end{fulllineitems}

\index{set\_range() (in module OBM.Preprocessing)@\spxentry{set\_range()}\spxextra{in module OBM.Preprocessing}}

\begin{fulllineitems}
\phantomsection\label{\detokenize{OBM:OBM.Preprocessing.set_range}}\pysiglinewithargsret{\sphinxcode{\sphinxupquote{OBM.Preprocessing.}}\sphinxbfcode{\sphinxupquote{set\_range}}}{\emph{signal}, \emph{Range\_min=50}, \emph{Range\_max=100}}{}
Range function. Remove values lower than 50 or greater than 100, considered as non-physiological
\begin{quote}\begin{description}
\item[{Parameters}] \leavevmode
\sphinxstyleliteralstrong{\sphinxupquote{signal}} \textendash{} 1-d array, of shape (N,) where N is the length of the signal

\item[{Returns}] \leavevmode
preprocessed signal, 1-d numpy array.

\end{description}\end{quote}

\end{fulllineitems}



\chapter{Indices and tables}
\label{\detokenize{index:indices-and-tables}}\begin{itemize}
\item {} 
\DUrole{xref,std,std-ref}{genindex}

\item {} 
\DUrole{xref,std,std-ref}{modindex}

\item {} 
\DUrole{xref,std,std-ref}{search}

\end{itemize}


\renewcommand{\indexname}{Python Module Index}
\begin{sphinxtheindex}
\let\bigletter\sphinxstyleindexlettergroup
\bigletter{o}
\item\relax\sphinxstyleindexentry{OBM}\sphinxstyleindexpageref{OBM:\detokenize{module-OBM}}
\item\relax\sphinxstyleindexentry{OBM.ComplexityMeasures}\sphinxstyleindexpageref{OBM:\detokenize{module-OBM.ComplexityMeasures}}
\item\relax\sphinxstyleindexentry{OBM.DesaturationsMeasures}\sphinxstyleindexpageref{OBM:\detokenize{module-OBM.DesaturationsMeasures}}
\item\relax\sphinxstyleindexentry{OBM.HypoxicBurdenMeasures}\sphinxstyleindexpageref{OBM:\detokenize{module-OBM.HypoxicBurdenMeasures}}
\item\relax\sphinxstyleindexentry{OBM.ODIMeasure}\sphinxstyleindexpageref{OBM:\detokenize{module-OBM.ODIMeasure}}
\item\relax\sphinxstyleindexentry{OBM.OverallGeneralMeasures}\sphinxstyleindexpageref{OBM:\detokenize{module-OBM.OverallGeneralMeasures}}
\item\relax\sphinxstyleindexentry{OBM.PeriodicityMeasures}\sphinxstyleindexpageref{OBM:\detokenize{module-OBM.PeriodicityMeasures}}
\item\relax\sphinxstyleindexentry{OBM.Preprocessing}\sphinxstyleindexpageref{OBM:\detokenize{module-OBM.Preprocessing}}
\end{sphinxtheindex}

\renewcommand{\indexname}{Index}
\printindex
\end{document}